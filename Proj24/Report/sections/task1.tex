\chapter{An analytical starter}

\resp{Vicentini Gioele}

\section{Introduction: Some concepts to understand the paper}
Ideas in the paper are developed in the generating function formalism: $G(x) = \sum_{k=0}^{+\infty} x^k P(k)$. The probability of finding an edge which does not connect to the GCC is $u$, generated by $H_1(x)$.

As seen in class, by definition this probability can be obtained in the following way: for a generic edge exiting from a node with degree $k$, there is a probability $1 - p_2$ for it not to be there and a probability $p_2 H_1(x) ^{k'} q(k')$ that the edge exists and that the neighbors of the connected node are not in the GCC. $q(k)$ is the excess degree distribution $q(k) = kP(k)/\langle k\rangle$ which gives us the probability of having a neighbor of degree $k'$. Summing over all possible values of $k'$ gives:
\begin{align}
    H_1(x) &= (1-p_2) + \sum_{k'}p_2 H_1(x) ^{k'} q(k') \\
    &= (1-p_2) + \sum_{k'}p_2 H_1(x) ^{k'} k'P(k')/\langle k\rangle \\
    &= (1-p_2) + p_2 xG_1(H_1(x))
\end{align}
where we used the sencond order gen. function $xG_1(x) = \sum_{k}kP(k) = xG'_0(x)/G'_0(1)$.

This leaves us with $u = H_1(1) = (1-p_2) + p_2 xG_1(u)$, the usual equation in open form.
For the probability of finding a node not connected to the GCC, we proceed similarly: this node has degree $k$ with prob. $P(k)$ and it's not connected to the GCC if all its edges are not: $H_0(1) = \sum_{k} P(k)H_1(1)^k = G_0(H_1(1))$ and $H_0(x) = xG_0(H_1(x))$. This means that the probability of a node to be in the GCC is $S = 1 - H_0(1)$.

\section{The contribution of q-swapped cycles}
The previous formulas change after the q-swap operations since they change in a relevant way the topology. Assuming node degrees to be uncorrelated:
\begin{equation}
    \tilde{H}_1(x) = 1-p_2 + p_2\tilde{H}_1(x)P_1(1) + \sum_{q\geq 2}\left[ p_2(1-\Pi_q) \tilde{H}_1(x)^q P_1(q) + \Pi_q P_1(q) C_q(x)\right]
\end{equation}
which accounts for (from left to right):
\begin{itemize}
    \item the probability $1-p_2$ of not having an edge
    \item the prob. of having an edge to a neighbor of degree $1$, for which there is no possible q-swap
    \item the prob. (summed over degrees $q$ of the existing neighbor) of not applying q-swap and the neighbor is not in GCC
    \item the prob. of a neighbor over which we apply the q-swapping operation. $C_q(x)$ generates the prob. of not being in the GCC for the new q-swapped cycle.
\end{itemize}
In particular, since each edge of the cycle is occupied with prob. $p$ (SCP probability):
\begin{equation}
    C_q(x) = qp^{q-1}(1-p)xG_1(\tilde{H}_1(x))^{q} + p^q xG_1(\tilde{H}_1(x))^{q} \sum_{l=0}^{q-2} \left[ (l + 1)p^l xG_1(\tilde{H}_1(x))^{q} (1-p)^{2-l} \right]
\end{equation}
note that if an edge connects us to a q-swapped cycle and two or more edges of this cycle are not occupied (SCP probability being $p < 1$), we have to account in different ways for the cases in which some of the edges are missing (otherwise we count the probability of, for example, having some nodes which do not belong to the GCC but not having a connection to them). if $l$ is the number of occupied edges, this means that $l=q$ has only one disposition of the edges, $l=q-1$ has $q$, $l = 0$ up to $l = q-2$ has $l+1$, since $l$ nodes must be connected for us to count properly the probability of not being connected to the GCC. The shape of $C_q(x)$ is connected to this. In the paper they write $(l+1)p^l(1-p)^2\dots$ instead of the binomial result $(l+1)p^l(1-p)^{2-l}\dots$, but i'm almost sure it's a typo on their side.

For $\tilde{H}_0(x)$, instead, denoting with $\mu_q$ the probability for a node of degree $q$ of undergoing q-swap:
\begin{equation}
    \tilde{H}_0(x) = P(0) + \tilde{H}_1(x)P(1) + \sum_{q\geq 2}\left[ \Pi_q \mu_q P(q) + (1 - \Pi_q \mu_q) P(q) \tilde{H}_1(x)^q\right]
\end{equation}
which are again the probabilities for degrees $0, 1$ which cannot be q-swapped, the prob. of having a q-swap which sets the degree of the node to zero (thus, not in GCC), the prob. of not having a q-swap and not being connected to GCC. In the paper they split the sum in order to obtain $xG_0(\tilde{H_1(x)})$.

\newpage